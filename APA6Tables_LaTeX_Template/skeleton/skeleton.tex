\documentclass[]{article}
\usepackage{lmodern}
\usepackage{amssymb,amsmath}
\usepackage{ifxetex,ifluatex}
\usepackage{fixltx2e} % provides \textsubscript
\ifnum 0\ifxetex 1\fi\ifluatex 1\fi=0 % if pdftex
  \usepackage[T1]{fontenc}
  \usepackage[utf8]{inputenc}
\else % if luatex or xelatex
  \ifxetex
    \usepackage{mathspec}
    \usepackage{xltxtra,xunicode}
  \else
    \usepackage{fontspec}
  \fi
  \defaultfontfeatures{Mapping=tex-text,Scale=MatchLowercase}
  \newcommand{\euro}{€}
\fi
% use upquote if available, for straight quotes in verbatim environments
\IfFileExists{upquote.sty}{\usepackage{upquote}}{}
% use microtype if available
\IfFileExists{microtype.sty}{%
\usepackage{microtype}
\UseMicrotypeSet[protrusion]{basicmath} % disable protrusion for tt fonts
}{}
\usepackage[margin=1in]{geometry}
\ifxetex
  \usepackage[setpagesize=false, % page size defined by xetex
              unicode=false, % unicode breaks when used with xetex
              xetex]{hyperref}
\else
  \usepackage[unicode=true]{hyperref}
\fi
\hypersetup{breaklinks=true,
            bookmarks=true,
            pdfauthor={Author Name},
            pdftitle={Title},
            colorlinks=true,
            citecolor=blue,
            urlcolor=blue,
            linkcolor=magenta,
            pdfborder={0 0 0}}
\urlstyle{same}  % don't use monospace font for urls
\usepackage{graphicx,grffile}
\makeatletter
\def\maxwidth{\ifdim\Gin@nat@width>\linewidth\linewidth\else\Gin@nat@width\fi}
\def\maxheight{\ifdim\Gin@nat@height>\textheight\textheight\else\Gin@nat@height\fi}
\makeatother
% Scale images if necessary, so that they will not overflow the page
% margins by default, and it is still possible to overwrite the defaults
% using explicit options in \includegraphics[width, height, ...]{}
\setkeys{Gin}{width=\maxwidth,height=\maxheight,keepaspectratio}
\setlength{\parindent}{0pt}
\setlength{\parskip}{6pt plus 2pt minus 1pt}
\setlength{\emergencystretch}{3em}  % prevent overfull lines
\providecommand{\tightlist}{%
  \setlength{\itemsep}{0pt}\setlength{\parskip}{0pt}}
\setcounter{secnumdepth}{0}

%%% Use protect on footnotes to avoid problems with footnotes in titles
\let\rmarkdownfootnote\footnote%
\def\footnote{\protect\rmarkdownfootnote}

%%% Change title format to be more compact
\usepackage{titling}

% Create subtitle command for use in maketitle
\newcommand{\subtitle}[1]{
  \posttitle{
    \begin{center}\large#1\end{center}
    }
}

\setlength{\droptitle}{-2em}
  \title{Title}
  \pretitle{\vspace{\droptitle}\centering\huge}
  \posttitle{\par}
  \author{Author Name}
  \preauthor{\centering\large\emph}
  \postauthor{\par}
  \date{}
  \predate{}\postdate{}

\usepackage[singlelinecheck=false]{caption}
\usepackage{graphicx}
\usepackage{booktabs}
\usepackage{tabularx}
\usepackage{mathptmx}
\usepackage{pdflscape}
\usepackage{dcolumn}
\DeclareMathVersion{nxbold}
\SetSymbolFont{operators}{nxbold}{OT1}{cmr} {b}{n}
\SetSymbolFont{letters}  {nxbold}{OML}{cmm} {b}{it}
\SetSymbolFont{symbols}  {nxbold}{OMS}{cmsy}{b}{n}
\makeatletter
\newcolumntype{B}[3]{>{\mathversion{nxbold}\DC@{#1}{#2}{#3} }c<{\DC@end} }
\makeatother

% Redefines (sub)paragraphs to behave more like sections
\ifx\paragraph\undefined\else
\let\oldparagraph\paragraph
\renewcommand{\paragraph}[1]{\oldparagraph{#1}\mbox{}}
\fi
\ifx\subparagraph\undefined\else
\let\oldsubparagraph\subparagraph
\renewcommand{\subparagraph}[1]{\oldsubparagraph{#1}\mbox{}}
\fi

\begin{document}
\maketitle

\newcolumntype{d}{D{.}{.}{0.2}}

\begin{table}[h]
\caption{Correlation Table of Dataset}
\begin{tabularx}{1\textwidth}{r@{.~}D{.}{.}{3.2}D{.}{.}{1.2} * {14}{d}}
\toprule
\multicolumn{1}{r}{} & \multicolumn{1}{c}{Mean} & \multicolumn{1}{c}{SD} & \multicolumn{1}{>{\centering\arraybackslash}X}{1} & \multicolumn{1}{>{\centering\arraybackslash}X}{2} & \multicolumn{1}{>{\centering\arraybackslash}X}{3} & \multicolumn{1}{>{\centering\arraybackslash}X}{4} & \multicolumn{1}{>{\centering\arraybackslash}X}{5} & \multicolumn{1}{>{\centering\arraybackslash}X}{6} & \multicolumn{1}{>{\centering\arraybackslash}X}{7} & \multicolumn{1}{>{\centering\arraybackslash}X}{8} & \multicolumn{1}{>{\centering\arraybackslash}X}{9} & \multicolumn{1}{>{\centering\arraybackslash}X}{10} & \multicolumn{1}{>{\centering\arraybackslash}X}{11} & \multicolumn{1}{>{\centering\arraybackslash}X}{12} & \multicolumn{1}{>{\centering\arraybackslash}X}{13} & \multicolumn{1}{>{\centering\arraybackslash}X}{14}\\
\toprule 1 & -0.02 & 0.97 &  &  &  &  &  &  &  &  &  &  &  &  &  & \\
2 & 0.01 & 1.02 & .04 &  &  &  &  &  &  &  &  &  &  &  &  & \\
3 & 0.01 & 0.98 & .02 & -.01 &  &  &  &  &  &  &  &  &  &  &  & \\
4 & 0.03 & 0.98 & -.02 & .01 & -.03 &  &  &  &  &  &  &  &  &  &  & \\
5 & 0.04 & 1.00 & .01 & .01 & -.00 & -.02 &  &  &  &  &  &  &  &  &  & \\
6 & -0.02 & 1.00 & -.03 & .03 & .01 & .06 & -.03 &  &  &  &  &  &  &  &  & \\
7 & -0.00 & 0.98 & .03 & -.02 & .05 & -.04 & .01 & .00 &  &  &  &  &  &  &  & \\
8 & 0.04 & 1.01 & -.00 & -.04 & .02 & .04 & .03 & .03 & -.06 &  &  &  &  &  &  & \\
9 & 0.08 & 0.98 & .01 & -.04 & -.01 & -.01 & .06 & .04 & -.02 & -.00 &  &  &  &  &  & \\
10 & 0.04 & 0.99 & -.03 & .01 & -.03 & -.03 & -.02 & .05 & -.00 & -.00 & .03 &  &  &  &  & \\
11 & -0.01 & 1.00 & -.05 & -.02 & -.02 & -.02 & -.03 & -.02 & .02 & .02 & .02 & -.04 &  &  &  & \\
12 & -0.04 & 1.01 & -.01 & .06 & .01 & -.01 & .04 & -.03 & .00 & .01 & -.01 & -.03 & -.03 &  &  & \\
13 & 0.00 & 1.05 & -.05 & \multicolumn{1}{B{.}{.}{0.2}}{.07} & .05 & -.01 & -.02 & -.05 & .00 & -.03 & -.01 & -.01 & -.02 & \multicolumn{1}{B{.}{.}{0.2}}{.07} &  & \\
14 & 0.00 & 1.00 & -.02 & -.02 & .01 & .03 & -.03 & .06 & .01 & \multicolumn{1}{B{.}{.}{0.2}}{.07} & -.01 & -.06 & .04 & .01 & -.06 & \\
15 & -0.01 & 1.02 & .00 & .01 & -.02 & -.02 & -.01 & .01 & -.03 & -.00 & -.05 & .05 & .01 & .04 & .05 & -.05\\
\toprule
\end{tabularx}

\emph{Note}: Significant correlations are bolded at $\alpha = .05$
\end{table}

\end{document}
